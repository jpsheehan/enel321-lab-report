% !TEX TS-program = pdflatex
% !TEX encoding = UTF-8 Unicode

% This is a simple template for a LaTeX document using the "article" class.
% See "book", "report", "letter" for other types of document.

\documentclass[12pt]{article} % use larger type; default would be 10pt

\usepackage[utf8]{inputenc} % set input encoding (not needed with XeLaTeX)

%%% Examples of Article customizations
% These packages are optional, depending whether you want the features they provide.
% See the LaTeX Companion or other references for full information.

%%% PAGE DIMENSIONS
\usepackage{geometry} % to change the page dimensions
\geometry{a4paper} % or letterpaper (US) or a5paper or....
% \geometry{margin=2in} % for example, change the margins to 2 inches all round
% \geometry{landscape} % set up the page for landscape
%   read geometry.pdf for detailed page layout information

\usepackage{graphicx} % support the \includegraphics command and options

%%% PACKAGES
\usepackage{booktabs} % for much better looking tables
\usepackage{array} % for better arrays (eg matrices) in maths
\usepackage{paralist} % very flexible & customisable lists (eg. enumerate/itemize, etc.)
\usepackage{verbatim} % adds environment for commenting out blocks of text & for better verbatim
\usepackage{subfig} % make it possible to include more than one captioned figure/table in a single float
% These packages are all incorporated in the memoir class to one degree or another...
\usepackage{pdfpages}
\usepackage{amsmath,mathtools,amsfonts}
\usepackage{bm}
\usepackage{blindtext}

%%% HEADERS & FOOTERS
\usepackage{fancyhdr} % This should be set AFTER setting up the page geometry
\pagestyle{fancy} % options: empty , plain , fancy
\renewcommand{\headrulewidth}{1pt} % customise the layout...
\lhead{Student ID: 53366509}\chead{Jesse Sheehan}\rhead{jps111@uclive.ac.nz}
\lfoot{}\cfoot{\thepage}\rfoot{}

%%% END Article customizations

\title{ENEL321 Lab Report}
\date{\today}
\author{Jesse Patrick Sheehan}

\begin{document}

\maketitle

\vfill

\renewcommand{\abstractname}{Executive Summary}

\begin{abstract}

% Motivation, what was done, summary of results (very brief)

\blindtext

\end{abstract}

\newpage

\section{Methodology}

% Briefly introduce the problem, and experiment but don't derive the model.
% Explain how you designed your gains, and what values you used and why.
% Explain the different controllers used.

\blindtext

\blindtext

\blindtext

\newpage

\section{Results}

% Present experimental results.

\blindtext

\blindtext

\blindtext

\newpage

\section{Discussion}

% Compare results with model, explain differences.
% Focus on the why rather than the what.

\blindtext

\blindtext

\blindtext

\newpage

\section{Conclusion}

% Summarize what worked and what didn't, what could be improved as well as any limitations in the methods observed.

\blindtext

\blindtext

\blindtext

\end{document}
